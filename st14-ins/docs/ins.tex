\documentclass[12pt, twocolumn]{article}

% 引入相关的包
\usepackage{amsmath, listings, fontspec, geometry, graphicx, ctex, color, subfigure}

% 设定页面的尺寸和比例
\geometry{left = 1.5cm, right = 1.5cm, top = 1.5cm, bottom = 1.5cm}

% 设定两栏之间的间距
\setlength\columnsep{1cm}

% 设定字体,为代码的插入作准备
\newfontfamily\ubuntu{Ubuntu Mono}

% 头部信息
\title{\normf{这是标题}}
\author{\normf{陈烁龙}}
\date{\normf{\today}}

% 代码块的风格设定
\lstset{
	language=C++,
	basicstyle=\scriptsize\ubuntu,
	keywordstyle=\textbf,
	stringstyle=\itshape,
	commentstyle=\itshape,
	numberstyle=\scriptsize\ubuntu,
	showstringspaces=false,
	numbers=left,
	numbersep=8pt,
	tabsize=2,
	frame=single,
	framerule=1pt,
	columns=fullflexible,
	breaklines,
	frame=shadowbox, 
	backgroundcolor=\color[rgb]{0.97,0.97,0.97}
}

% 字体族的定义
\newcommand{\normf}{\kaishu}
\newcommand{\boldf}{\heiti}
\newcommand\keywords[1]{\boldf{关键词:} \normf #1}

\begin{document}
	
	% 插入头部信息
	\maketitle
	% 换页
	\thispagestyle{empty}
	\clearpage
	
	% 插入目录、图、表并换页
	\tableofcontents
	\listoffigures
	\listoftables
	\setcounter{page}{1}
	% 罗马字母形式的页码
	\pagenumbering{roman}
	\clearpage
	% 从该页开始计数
	\setcounter{page}{1}
	% 阿拉伯数字形式的页码
	\pagenumbering{arabic}
	
	\section{\normf{前言}}
	\normf
	记录惯导课程的学习!
	\section{\normf{惯性导航和惯性器件}}
	导航是指:运动物体随时间变化的位置(Position)、速度(Velocity)和姿态(Attitude),这三者构成了导航姿态(PVA)。导航主要有两种原理:
	\begin{enumerate}
		\item 航位推算(Dead Reckoning):通过推断连续帧间导航状态的变换量得到航迹,如视觉里程计(VO)、惯导(INS);
		\item 直接定位(Direct Fixing):直接获得运动物体在参考坐标系下的导航状态,如卫导(GNSS)。
	\end{enumerate}
	
	对于惯导而言,借助加速度计和陀螺仪测定载体相对于惯性空间的加速度和角速度。具体来说:
	
	\begin{enumerate}
		\item 加速度计(Accelerometer):
		测量相对于惯性参考系的加速度(即比力)的传感器。公式:
		\begin{eqnarray*}
			f=a-g
		\end{eqnarray*}
	其中:$f$即为比力,为加速度计的直接输出;$a$为载体相对于惯性参考空间的运动加速度;$g$为万有引力加速度\footnote{\normf{注意:加速度计的直接输出是物体运动加速度减去万有引力加速度,是减号,不是加号。}};
		
		\item 陀螺仪(Gyroscope):测量相对于惯性参考系的角速率的传感器。是惯导系统中最为关键的器件。
		
		
	\end{enumerate}
	将二者结合在一起,就得到了惯性测量单元(IMU),其是由一个三轴加速度计和一个三轴陀螺构成。将惯性导航算法与IMU结合,就得到了惯性导航系统(INS)。
	
	惯性导航系统有两类:平台式系统和捷联式系统。前者通过闭环的控制系统感知IMU中陀螺仪的输出变化来保证IMU中的加速度计的朝向始终保持一致。此种系统对于算法要求较低,精度较高,但是十分昂贵,一般只使用到高精度的场景下。而后者则通过复杂的算法来计算运动物体的导航状态,而不在要求要保证IMU中的加速度计的朝向始终保持一致。该类系统价格便宜,精度可行但不及平台式系统,但是应用广泛。
	
	\section{\normf{惯性器件的误差和标定}}
	考虑这样一个场景:假设有一辆车在地表行驶,海拔高为$h$,其载体里的IMU坐标系统$xyz$与当地的导航坐标系$NED$对齐。该辆小车的北向速度为$v_N$,东向速度为$v_E$,则该小车上的IMU输出为多少?
	
	很明显,由于:
	\begin{eqnarray*}
		\begin{aligned}
			&\to\dot{\varphi}=\frac{v_N}{R+h}\\
			&\to\dot{\lambda}=\frac{v_E}{(R+h)\cos\varphi}
		\end{aligned}
	\end{eqnarray*}
	所以,陀螺仪的输出为:
	\begin{eqnarray*}
	\begin{aligned}
	\to \omega_E&=-\dot{\varphi}=-\frac{v_N}{R+h}\\
	\to\omega_N&=\dot{\lambda}\cos\varphi+\omega_e\cos\varphi\\
	&=\frac{v_E}{R+h}+\omega_e\cos\varphi\\
	\to\omega_D&=-\dot{\lambda}\sin\varphi-\omega_e\sin\varphi\\
	&=-\frac{v_E}{R+h}\tan\varphi-omega_e\sin\varphi
	\end{aligned}
	\end{eqnarray*}

	接下来介绍常用的坐标系:
	\begin{enumerate}
		\item 惯性坐标系(i-frame):原点为地球质心,z轴沿地球自转轴指向协议地极,x轴指向春分点;
		\item 地固坐标系(e-frame):和地球固连在一起,x轴指向赤道与本初子午线的交点;
		\item 导航坐标系(n-frame):x轴沿参考椭球北向,y轴沿参考参考椭球东向,z轴沿椭球法线向下;
		\item IMU坐标系(载体坐标系b-frame):原点为IMU中心,x轴为陀螺正前方,y轴为陀螺右方,z轴为陀螺下方,即"xyz"对应"前右下"。
		
	\end{enumerate}

	陀螺的测量值为:
	\begin{equation*}
		\widetilde{\omega}=\omega+b_\omega+S_\omega+N\omega+\varepsilon_\omega
	\end{equation*}
	其中:$\widetilde{\omega}$为实际陀螺的输出,$\omega$为真实的角速度,$b_\omega$为陀螺的零偏,$S$为陀螺的比例因子误差矩阵,$N$为陀螺交轴耦合误差矩阵,$\varepsilon_\omega$为陀螺噪声矢量。且有:
	\begin{equation*}
		b_\omega=\begin{pmatrix}
			b_{\omega,x}\\
			b_{\omega,y}\\
			b_{\omega,z}
		\end{pmatrix},
			S=\begin{pmatrix}
		S_x&0&0\\
		0&S_y&0\\
		0&0&S_z
		\end{pmatrix}
	\end{equation*}
	
	\begin{equation*}
				N=\begin{pmatrix}
			0&\gamma_{xy}&\gamma_{xz}\\
			\gamma_{yx}&0&\gamma_{yz}\\
			\gamma_{zx}&\gamma_{zy}&0
		\end{pmatrix}
	\end{equation*}
	
	加速度计的测量值为:
	\begin{equation*}
		\widetilde{f}=f+b_f+S_1f+S_2f^2+Nf+\delta g+\varepsilon_f
	\end{equation*}
	其中:$\widetilde{f}$为实际加速度计的输出,$f$为真实的角速度,$b_f$为加速度计的零偏,$S_1$为加速度计的线性比例因子误差矩阵,$S_2$为加速度计的非线性比例因子误差矩阵,$N$为加速度计交轴耦合误差矩阵,$\delta g$为重力异常,$\varepsilon_f$为加速度计噪声矢量。
	
	\section{\normf{姿态更新}}
	以四元素为例,问题为已知:
	\begin{enumerate}
		\item $\boldsymbol{q}_b^n(t_{k-1})$,前一时刻b系相对于n系的姿态;
		\item $\Delta \boldsymbol{\theta}_k$,$\Delta \boldsymbol{\theta}_{k-1}$,当前时刻和前一时刻的陀螺角增量输出,即:
		\begin{equation*}
			\Delta \boldsymbol{\theta}_k=\int_{t_{k-1}}^{t_k}\boldsymbol{\omega}_{ib}^b(t)dt
		\end{equation*}
		\item $\boldsymbol{\omega}_{ie}^n$,地球自转角速率在n系下的投影,$\boldsymbol{\omega}_{en}^n$,n系相对于e系的角速率在n系下的投影:
		\begin{equation*}
			\to\boldsymbol{\omega}_{ie}^n=\begin{pmatrix}
				 \omega_e\cos\varphi&0&-\omega_e\sin\varphi
			\end{pmatrix}^T
		\end{equation*}
		\begin{equation*}
			\to\boldsymbol{\omega}_{en}^n=\begin{pmatrix}
				 \frac{v_E}{R_N+h}&-\frac{v_N}{R_M+h}&-\frac{v_E\tan\varphi}{R_N+h}
			\end{pmatrix}^T
		\end{equation*}
	\end{enumerate}
	求解:
	\begin{enumerate}
		\item $\boldsymbol{q}_b^n(t_k)$,当前时刻b系相对于n系的姿态。
	\end{enumerate}

	以四元素为例:姿态递推计算的运算式子为:
	\begin{equation*}
		\boldsymbol{q}_b^n(t_k)=\boldsymbol{q}_{n(k-1)}^{n(k)}\otimes \boldsymbol{q}_{b(k-1)}^{n(k-1)}\otimes \boldsymbol{q}_{b(k)}^{b(k-1)}
	\end{equation*}
	其中:
	\begin{enumerate}
		\item 使用等效旋转矢量更新b系从$t_{k-1}$到$t_k$时刻的姿态变换(最后一项):
		\begin{equation*}
			\to \boldsymbol{\phi}_k=\Delta \boldsymbol{\theta}_k+\frac{1}{12}\Delta \boldsymbol{\theta}_{k-1}\times \Delta\boldsymbol{\theta}_k
		\end{equation*}
		\begin{equation*}
			\to \boldsymbol{q}_{b(k)}^{b(k-1)}=\begin{pmatrix}
				\cos{\Vert\frac{1}{2}\boldsymbol{\phi}_k\Vert}\\
				\sin{\Vert\frac{1}{2}\boldsymbol{\phi}_k\Vert}\frac{\boldsymbol{\phi}_k}{\Vert\boldsymbol{\phi}_k\Vert}
			\end{pmatrix}
		\end{equation*}
		\item 使用等效旋转矢量更新n系从$t_{k-1}$到$t_k$时刻的姿态变换(第一项):
		\begin{equation*}
			\to\boldsymbol{\zeta}_k=[\boldsymbol{\omega}_{ie}^n(t_{k-1})+\boldsymbol{\omega}_{en}^n(t_{k-1})]\Delta t
		\end{equation*}
			\begin{equation*}
		\to \boldsymbol{q}_{n(k)}^{n(k-1)}=\begin{pmatrix}
			\cos{\Vert\frac{1}{2}\boldsymbol{\zeta}_k\Vert}\\
			\sin{\Vert\frac{1}{2}\boldsymbol{\zeta}_k\Vert}\frac{\boldsymbol{\zeta}_k}{\Vert\boldsymbol{\zeta}_k\Vert}
		\end{pmatrix}
	\end{equation*}
	\item 带入公式进行递推,最后对得到的结果姿态四元素进行归一化。
	\end{enumerate}

	\section{\normf{速度更新算法}}
	总的速度更新算法公式:
	\begin{equation*}
		\boldsymbol{v}_k^n=\boldsymbol{v}_{k-1}^n+\Delta\boldsymbol{v}_{f,k}^n+\Delta\boldsymbol{v}_{g/cor,k}^n
	\end{equation*}
	其中公式左边为当前时刻载体在n系的速度,公式右边第一项为上一时刻载体在n系的速度,中间项为比力积分项,最后一项为重力/哥氏积分项。
	\begin{enumerate}
		\item 计算重力/哥氏积分项:
		\begin{equation*}
			\begin{aligned}
			&\Delta\boldsymbol{v}_{g/cor,k}^n=\\&
			[\boldsymbol{g}_l^n-(2\boldsymbol{\omega}_{ie}^n+\boldsymbol{\omega}_{en}^n)\times\boldsymbol{v}_e^n]_{t_{k-1/2}}(t_{k}-t_{k-1})	
			\end{aligned}
		\end{equation*}
	其中,$\boldsymbol{g}_l^n$为当地重力加速度在导航系下的投影。
	\item 计算比力积分项:
	\begin{equation*}
		\Delta\boldsymbol{v}_{f,k}^n=[\boldsymbol{I}-(\frac{1}{2}\boldsymbol{\zeta}_{k-1,k}\times)]\boldsymbol{C}_{b(k-1)}^{n(k-1)}\Delta \boldsymbol{v}_{f,k}^{b(k-1)}
	\end{equation*}
	\begin{equation*}
		\boldsymbol{\zeta}_{k-1,k}=[\boldsymbol{\omega}_{en}^n(t_{k-1/2})+\boldsymbol{\omega}_{ie}^n(t_{k-1/2})](t_k-t_{k-1})
	\end{equation*}
	\begin{equation*}
		\begin{aligned}
				\Delta \boldsymbol{v}_{f,k}^{b(k-1)}=&\Delta\boldsymbol{v}_k+\frac{1}{2}\Delta\boldsymbol{\theta}_k\times\Delta\boldsymbol{v}_k+\\
				&\frac{1}{12}(\Delta\boldsymbol{\theta}_{k-1}\times\Delta\boldsymbol{v}_k+\Delta\boldsymbol{v}_{k-1}\times\Delta\boldsymbol{\theta}_k)
		\end{aligned}
	\end{equation*}
	需要注意的是,我们求解当前时刻的速度向量时,用到了上一时刻和当前时刻的速度来求解中间量(求解重力/哥氏积分项和计算比力积分项当中都出现了)。对于这种,我们使用线性递推即可(即使用$t_{k-2}$和$t_{k-1}$的状态来推测$t_k$的状态,线性递推是最简单的方式)。
	\end{enumerate}
	\section{\normf{位置更新}}
	如果令:
	\begin{equation*}
		\begin{aligned}
			\bar{h}&=\frac{1}{2}(h_k+h_{k-1})\\
			\bar{\varphi}&=\frac{1}{2}(\varphi_k+\varphi_{k-1})\\
			v_{N,k-1/2}&=\frac{1}{2}(v_{N,k}+v_{N,k-1})\\
			v_{E,k-1/2}&=\frac{1}{2}(v_{E,k}+v_{E,k-1})\\
			v_{D,k-1/2}&=\frac{1}{2}(v_{D,k}+v_{D,k-1})
		\end{aligned}
	\end{equation*}
	那么位置更新的公式为:
	\begin{equation*}
		\begin{aligned}
			\to h_k&=h_{k-1}-v_{D,k-1/2}(t_k-t_{k-1})\\
			\to \varphi_k&=\varphi_{k-1}+\frac{v_{N,k-1/2}}{R_{M,k-1/2}+\bar{h}}(t_k-t_{k-1})\\
			\to \lambda_k&=\lambda_{k-1}+\frac{v_{E,k-1/2}}{(R_{N,k-1/2}+\bar{h})\cos\bar{\varphi}}(t_k-t_{k-1})
		\end{aligned}
	\end{equation*}

	\section{\normf{误差模型}}
	速度的误差模型为:
	\begin{equation*}
		\begin{aligned}
					\delta \boldsymbol{\dot{v}}^n_{eb}&=\boldsymbol{C}_b^n\delta\boldsymbol{f}^b+\boldsymbol{f}^n\times\delta\boldsymbol{\phi}_{b}^n\\
					&-(2\boldsymbol{\omega}_{ie}^n+\boldsymbol{\omega}_{en}^n)\times\delta\boldsymbol{v}^n_{eb}\\
					&+\boldsymbol{v}^n_{eb}\times(2\delta\boldsymbol{\omega}_{ie}^n+\delta\boldsymbol{\omega}_{en}^n)+\delta\boldsymbol{g}_p^n
		\end{aligned}
	\end{equation*}
	
	位置的误差模型为:
	\begin{equation*}
		\begin{aligned}
			\delta\boldsymbol{\dot{r}}^n_{eb}=-\boldsymbol{\omega}^n_{en}\times+\delta\boldsymbol{r}^n_{eb}+\delta\boldsymbol{\phi}^n_b\times\boldsymbol{v}^n_{eb}+\delta\boldsymbol{v}^n_{eb}
		\end{aligned}
	\end{equation*}
	
	姿态的误差模型为:
	\begin{equation*}
		\begin{aligned}
			\delta\boldsymbol{\dot{\phi}}_b^n=-\boldsymbol{\omega}_{in}^n\times\delta\boldsymbol{\phi}_b^n+\delta\boldsymbol{\omega}_{in}^n-\delta\boldsymbol{\omega}_{ib}^n
		\end{aligned}
	\end{equation*}
\end{document}

