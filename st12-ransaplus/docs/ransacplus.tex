\documentclass[12pt, twocolumn]{article}

% 引入相关的包
\usepackage{amsmath, listings, fontspec, geometry, graphicx, ctex, color, subfigure}

% 设定页面的尺寸和比例
\geometry{left = 1.5cm, right = 1.5cm, top = 1.5cm, bottom = 1.5cm}

% 设定两栏之间的间距
\setlength\columnsep{1cm}

% 设定字体,为代码的插入作准备
\newfontfamily\ubuntu{Ubuntu Mono}

% 头部信息
\title{\normf{这是标题}}
\author{\normf{陈烁龙}}
\date{\normf{\today}}

% 代码块的风格设定
\lstset{
	language=C++,
	basicstyle=\scriptsize\ubuntu,
	keywordstyle=\textbf,
	stringstyle=\itshape,
	commentstyle=\itshape,
	numberstyle=\scriptsize\ubuntu,
	showstringspaces=false,
	numbers=left,
	numbersep=8pt,
	tabsize=2,
	frame=single,
	framerule=1pt,
	columns=fullflexible,
	breaklines,
	frame=shadowbox, 
	backgroundcolor=\color[rgb]{0.97,0.97,0.97}
}

% 字体族的定义
\newcommand{\normf}{\kaishu}
\newcommand{\boldf}{\heiti}
\newcommand\keywords[1]{\boldf{关键词:} \normf #1}

\begin{document}
	
	% 插入头部信息
	\maketitle
	% 换页
	\thispagestyle{empty}
	\clearpage
	
	% 插入目录、图、表并换页
	\tableofcontents
	\listoffigures
	\listoftables
	\setcounter{page}{1}
	% 罗马字母形式的页码
	\pagenumbering{roman}
	\clearpage
	% 从该页开始计数
	\setcounter{page}{1}
	% 阿拉伯数字形式的页码
	\pagenumbering{arabic}
	
	\begin{abstract}
		\normf
		本文档演示了在\LaTeX 排版系统中一些常用的方法。当然,一些地方仍然没有比较好的解决方式,如图片、表格的跨栏插入等。另外,在后续的使用中,会基于新的需求进行新内容的引入。
		
		
	\end{abstract}
	
	\noindent\keywords{\LaTeX,模板,参考,初体验}
	
	\section{\normf{代码块的使用}}
	\normf
	
	接下来进行代码块的演示:
	\lstinputlisting[firstline = 12, lastline=20, caption=\normf{某个代码块},label=code1]{./code/main.cpp}
	
	代码块\ref{code1} 是一个和线程\footnote{\normf{是操作系统能够进行运算调度的最小单位。}}有关的代码块。列表\ref{code1}演示的是通过文件进行代码的插入,当然,也可以直接通过在\LaTeX 源代码中插入代码直接进行展示。如代码列表\ref{code2}所示:
	\begin{lstlisting}[label=code2,caption={\normf 直接插入源代码}]
	#include <iostream>
	
	int main() {
	// print some message to the console
	std::cout << "hello, world!" << std::endl;
	return 0;
	}
	\end{lstlisting}
	
	
	\section{\normf{数学公式的表达}}
	\normf
	
	设直角三角形的三边长度分别为$a,b,c$,其中$c$为直角边,则根据\textit{勾股定理}有:
	\begin{equation}
	a^2+b^2=c^2
	\end{equation}
	推广到任意三角形,有:
	\begin{equation}
	a^2+b^2-c^2=2ab\cos{C}
	\end{equation}
	其中$\angle C$是直角边$c$所对的角。
	
	来一个比较长一点的数学公式,如下公式所示:
	\begin{equation}
	% 为了让长公式能在一个栏里完全显示,所以需要对公式进行换行
	\begin{aligned}
	f(x)&=\sum_{i=0}^{n}\frac{1}{2}\kappa (e_{i}(\alpha_{j}))\\&=\sum_{i=0}^{n} 
	\frac{1}{2}\kappa(\hat{g_{i}}(\alpha_{j})-g_{i}(\alpha_{j}))	
	\end{aligned}
	\end{equation}
	
	其中$\kappa(x)$为核函数,用于对优化过程中的异常值造成的影响进行限制。
	
	当然,我们也可以输入方程组,比如如下的方程组描述了相机的径向畸变模型:
		\begin{equation}
	\begin{cases}
	x_{dist}=x(1+k_{1}r^{2}+k_{2}r^{4}+k_{3}r^{6}) \\
	y_{dist}=y(1+k_{1}r^{2}+k_{2}r^{4}+k_{3}r^{6}) 
	\end{cases}
	\end{equation}
	
	\section{\normf{图片的使用}}
	\normf
	
	下图\ref{fig:pic}为图片的插入演示。其中图\ref{fig:pic:android}为安卓操作系统的图标,图\ref{fig:pic:duo}为漫画作品“哆啦A梦”中的人物。而图\ref{fig:pic:ball}为旋转$90^{\circ}$后的热气球图片。
	\begin{figure}[htbp]
		\centering
		
		\subfigure[\normf{安卓图标}]{
			\centering
			\includegraphics[width=0.45\linewidth]{img/android.png}
			\label{fig:pic:android}
		}
		\subfigure[\normf{多拉A梦}]{
			\centering
			\includegraphics[width=0.45\linewidth]{img/duo.jpeg}
			\label{fig:pic:duo}
		}
		\subfigure[\normf{热气球}]{
			\centering
			\includegraphics[height=0.9\linewidth, angle=90]{img/ball.jpeg}
			\label{fig:pic:ball}
		}
		
		\caption{\normf{某组图片}}
		
		\label{fig:pic}
	\end{figure}
	
	\section{\normf{脚注}}
	\normf
	脚注其实非常简单。这\footnote{\normf{这是一个脚注}}就是一个简简单单、朴实无华的脚注。
	
	\section{\normf{列表的使用}}
	\normf
	列表可以分为“有序列表”和“无序列表”,下面进行演示:
	\begin{enumerate}
		\item 有序列表
		
		接下来要做的事情有:
		\begin{enumerate}
			\item 吃饭
			
			吃是一种文化,民以食为天,一般一日三餐,分吃早饭、午饭和晚饭,发展到现在,在许多大中城市,许多人形成了吃夜宵的习惯;
			\item 去图书馆
			
			图书馆,是搜集、整理、收藏图书资料以供人阅览、参考的机构;
			\item 跑步
			
			指陆生动物使用足部移动;
			\item 洗澡
			
			能清除汗垢油污、消除疲劳、舒筋活血、改善睡眠、提高皮肤新陈代谢功能和抗病力;
			\item 睡觉
			
			一般是指人类睡眠,是人类不可缺少的一种生理现象。
		\end{enumerate}
		\item 无序列表
		
		我会的编程语言有:
		\begin{itemize}
			\item $CPP$
			
			$C++$是一种计算机高级程序设计语言,由$C$语言扩展升级而产生;
			\item $Python$
			
			$Python$由荷兰数学和计算机科学研究学会的吉多·范罗苏姆 于$1990$年代初设计,作为一门叫做$ABC$语言的替代品;
			\item $Java$
			
			$Java$是一门面向对象的编程语言,不仅吸收了$C++$语言的各种优点,还摒弃了$C++$里难以理解的多继承、指针等概念,因此$Java$语言具有功能强大和简单易用两个特征
			
			\item $\cdots$
		\end{itemize}
	\end{enumerate}
	
	\section{\normf{论文的引用}}
	\normf
	众所周知,$ORB-SLAM$\cite{ref:orb}算法基于关键帧构建共视图,并在共视图的基础上构建了本质图。它们在重定位、回环检测等处都有重要的作用。
	
	\section{\normf{表格的使用}}
	\normf
	表格\ref{tab:tab}为示例的表格。这可以从下面的网站:$$https://www.tablesgenerator.com$$可直接获得源代码。
	
	%由于跨栏表格会跑到下一页,所以可以在上一页的地方提前插入一个表格
	\begin{table*}[htbp]
		\centering
		\begin{tabular}{ccl}
			\hline
			\textbf{\normf{名称}}     & \textbf{\normf{版本号}}&\textbf{\normf{地址}} \\ \hline
			\textit{libflags} & \textit{1.0.0}  &\textit{https://github.com/Unsigned-Long/flags.git}\\ 
			\textit{libcsv} & \textit{1.0.0}  &\textit{https://github.com/Unsigned-Long/CSV-Handler.git}\\ 
			\textit{libangle} & \textit{1.0.0}  &\textit{https://github.com/Unsigned-Long/Angle.git}\\ \hline
		\end{tabular}
		\caption{\normf{这是一个表格}}
		\label{tab:tab}
	\end{table*}
	
	\section{\normf{说明}}
	\normf
	本模板\footnote{\normf{或者说是一种参考文档}},是针对中文论文、笔记或报告的,但是同样适用于英文的排版布局。
	
	\begin{thebibliography}{99}
		\bibitem{ref:orb}Mur-Artal, Raul, Jose Maria Martinez Montiel, and Juan D. Tardos. "ORB-SLAM: a versatile and accurate monocular SLAM system." IEEE transactions on robotics 31.5 (2015): 1147-1163.
		\bibitem{ref:latex}Kopka, H., and P. W. Daly. "A Guide to \LaTeX--Document." (1995).
	\end{thebibliography}
	
\end{document}

