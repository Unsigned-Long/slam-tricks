\documentclass[12pt, onecolumn]{article}

% 引入相关的包
\usepackage{amsmath, listings, fontspec, geometry, graphicx, ctex, color, subfigure, amsfonts, amssymb}
\usepackage{multirow}
\usepackage[table,xcdraw]{xcolor}
\usepackage[ruled]{algorithm2e}
\usepackage[hidelinks]{hyperref}
\hypersetup{
	colorlinks=true,
	linkcolor=red,
	citecolor=red,
}

% 设定页面的尺寸和比例
\geometry{left = 1.5cm, right = 1.5cm, top = 1.5cm, bottom = 1.5cm}

% 设定两栏之间的间距
\setlength\columnsep{1cm}

% 设定字体,为代码的插入作准备
\newfontfamily\ubuntu{Ubuntu Mono}
\newfontfamily\consolas{Consolas}

% 头部信息
\title{\normf{基于本质矩阵的两视图初始化}}
\author{\normf{陈烁龙}}
\date{\normf{\today}}

% 代码块的风格设定
\lstset{
	language=C++,
	basicstyle=\scriptsize\ubuntu,
	keywordstyle=\textbf,
	stringstyle=\itshape,
	commentstyle=\itshape,
	numberstyle=\scriptsize\ubuntu,
	showstringspaces=false,
	numbers=left,
	numbersep=8pt,
	tabsize=2,
	frame=single,
	framerule=1pt,
	columns=fullflexible,
	breaklines,
	frame=shadowbox, 
	backgroundcolor=\color[rgb]{0.97,0.97,0.97}
}

% 字体族的定义
% \fangsong \songti \heiti \kaishu
\newcommand\normf{\fangsong}
\newcommand\boldf{\heiti}
\newcommand\keywords[1]{\boldf{关键词:} \normf #1}

\newcommand\liehat[1]{\left[ #1 \right]_\times}
\newcommand\lievee[1]{\left[ #1 \right]^\vee}
\newcommand\liehatvee[1]{\left[ #1 \right]^\vee_\times}

\newcommand\mlcomment[1]{\iffalse #1 \fi}
%\newcommand\mlcomment[1]{ #1 }

\begin{document}
	
	% 插入头部信息
	\maketitle
	% 换页
	\thispagestyle{empty}
	\clearpage
	
	% 插入目录、图、表并换页
	\pagenumbering{roman}
	\tableofcontents
	\newpage
	\listoffigures
	\newpage
	\listoftables
	% 罗马字母形式的页码
	
	\clearpage
	% 从该页开始计数
	\setcounter{page}{1}
	% 阿拉伯数字形式的页码
	\pagenumbering{arabic}
	
	\section{\normf{基础矩阵}}
	\normf
	假设有两帧影像$F_1$和$F_2$,通过特征提取和匹配,得到了特征点匹配对几集合$\mathcal{P}$,则对于第$i$个特征匹配$\mathcal{P}_i=\left\lbrace {^{F_1}\boldsymbol{f}_m},{^{F_2}\boldsymbol{f}_n}\right\rbrace $,有:
	\begin{equation}
	{^{F_1}d_m}\cdot\boldsymbol{K}^{-1}\cdot\begin{pmatrix}
	{^{F_1}\boldsymbol{f}_m}\\1
	\end{pmatrix}=
	{^{F_1}_{F_2}\boldsymbol{R}}\cdot{^{F_2}d_n}\cdot\boldsymbol{K}^{-1}\cdot\begin{pmatrix}
	{^{F_2}\boldsymbol{f}_n}\\1
	\end{pmatrix}+{^{F_1}\boldsymbol{p}_{F_2}}
	\end{equation}
	\begin{equation}
	\liehat{^{F_1}\boldsymbol{p}_{F_2}}\cdot{^{F_1}d_m}\cdot\boldsymbol{K}^{-1}\cdot\begin{pmatrix}
	{^{F_1}\boldsymbol{f}_m}\\1
	\end{pmatrix}=\liehat{^{F_1}\boldsymbol{p}_{F_2}}\cdot
	{^{F_1}_{F_2}\boldsymbol{R}}\cdot{^{F_2}d_n}\cdot\boldsymbol{K}^{-1}\cdot\begin{pmatrix}
	{^{F_2}\boldsymbol{f}_n}\\1
	\end{pmatrix}
	\end{equation}
	\begin{equation}
	0=\left( {^{F_1}d_m}\cdot\boldsymbol{K}^{-1}\cdot\begin{pmatrix}
	{^{F_1}\boldsymbol{f}_m}\\1
	\end{pmatrix}\right)^\top \cdot
	\liehat{^{F_1}\boldsymbol{p}_{F_2}}\cdot
	{^{F_1}_{F_2}\boldsymbol{R}}\cdot{^{F_2}d_n}\cdot\boldsymbol{K}^{-1}\cdot\begin{pmatrix}
	{^{F_2}\boldsymbol{f}_n}\\1
	\end{pmatrix}
	\end{equation}
	\begin{equation}
	0=\begin{pmatrix}
	{^{F_1}\boldsymbol{f}_m}\\1
	\end{pmatrix}^\top \cdot\boldsymbol{K}^{-\top}\cdot
	\liehat{^{F_1}\boldsymbol{p}_{F_2}}\cdot
	{^{F_1}_{F_2}\boldsymbol{R}}\cdot\boldsymbol{K}^{-1}\cdot\begin{pmatrix}
	{^{F_2}\boldsymbol{f}_n}\\1
	\end{pmatrix}
	\end{equation}
	得到:
	\begin{equation}
	{^{F_1}_{F_2}\boldsymbol{F}}=\boldsymbol{K}^{-\top}\cdot
	\liehat{^{F_1}\boldsymbol{p}_{F_2}}\cdot
	{^{F_1}_{F_2}\boldsymbol{R}}\cdot\boldsymbol{K}^{-1}=
	\begin{pmatrix}
	f_1&f_2&f_3\\
	f_4&f_5&f_6\\
	f_7&f_8&f_9
	\end{pmatrix}
	\end{equation}
	于是有:
	\begin{equation}
	\begin{pmatrix}
	{^{F_1}u_{m}}\\{^{F_1}v_{m}}\\1
	\end{pmatrix}^\top\cdot
	\begin{pmatrix}
	f_1&f_2&f_3\\
	f_4&f_5&f_6\\
	f_7&f_8&f_9
	\end{pmatrix}\cdot
	\begin{pmatrix}
	{^{F_2}u_{n}}\\{^{F_2}v_{n}}\\1
	\end{pmatrix}=0
	\end{equation}
	展开,得到:
	\begin{equation}
	{^{F_1}u_{m}}\cdot\left({^{F_2}u_{n}}\cdot f_1+{^{F_2}v_{n}}\cdot f_2+ f_3 \right) +
	{^{F_1}v_{m}}\cdot\left({^{F_2}u_{n}}\cdot f_4+{^{F_2}v_{n}}\cdot f_5+ f_6 \right) +
	\left({^{F_2}u_{n}}\cdot f_7+{^{F_2}v_{n}}\cdot f_8+ f_9 \right)=0
	\end{equation}
	也即:
	\begin{equation}
	\begin{pmatrix}
	{^{F_1}u_{m}}\cdot{^{F_2}u_{n}}\\{^{F_1}u_{m}}\cdot{^{F_2}v_{n}}\\{^{F_1}u_{m}}\\
	{^{F_1}v_{m}}\cdot{^{F_2}u_{n}}\\{^{F_1}v_{m}}\cdot{^{F_2}v_{n}}\\{^{F_1}v_{m}}\\
	{^{F_2}u_{n}}\\{^{F_2}v_{n}}\\1
	\end{pmatrix}^\top\cdot
	\begin{pmatrix}
	f_1\\f_2\\f_3\\
	f_4\\f_5\\f_6\\
	f_7\\f_8\\f_9
	\end{pmatrix}=0
	\end{equation}
	通过SVD分解求解参数向量,进而构造基础矩阵${^{F_1}_{F_2}\boldsymbol{F}}$。
	
	\section{\normf{三角化}}
	有关系式:
	\begin{equation}
	\begin{cases}
	\begin{aligned}
	{^{F_1}d_m}\cdot\begin{pmatrix}{^{F_1}\boldsymbol{f}_m}\\1\end{pmatrix}
	&=\boldsymbol{K}\cdot\begin{pmatrix}
	{^{W}_{F_1}\boldsymbol{R}}&{^{W}\boldsymbol{p}_{F_1}}
	\end{pmatrix}^\top\cdot\begin{pmatrix}
	{^{W}\boldsymbol{l}}\\1
	\end{pmatrix}={\boldsymbol{P}_1}\cdot\begin{pmatrix}
	{^{W}\boldsymbol{l}}\\1
	\end{pmatrix}
	\\
	{^{F_2}d_n}\cdot\begin{pmatrix}{^{F_2}\boldsymbol{f}_n}\\1\end{pmatrix}
	&=\boldsymbol{K}\cdot\begin{pmatrix}
	{^{W}_{F_2}\boldsymbol{R}}&{^{W}\boldsymbol{p}_{F_2}}
	\end{pmatrix}^\top\cdot\begin{pmatrix}
	{^{W}\boldsymbol{l}}\\1
	\end{pmatrix}={\boldsymbol{P}_2}\cdot\begin{pmatrix}
	{^{W}\boldsymbol{l}}\\1
	\end{pmatrix}
	\end{aligned}
	\end{cases}
	\end{equation}
	已知${\boldsymbol{P}_1}$和${\boldsymbol{P}_2}$,求解${^{W}\boldsymbol{l}}$。化简:
	\begin{equation}
	\begin{aligned}
	\boldsymbol{0}=\liehat{{^{F_1}d_m}\cdot\begin{pmatrix}{^{F_1}\boldsymbol{f}_m}\\1\end{pmatrix}}\cdot{^{F_1}d_m}\cdot\begin{pmatrix}{^{F_1}\boldsymbol{f}_m}\\1\end{pmatrix}&=\liehat{{^{F_1}d_m}\cdot\begin{pmatrix}{^{F_1}\boldsymbol{f}_m}\\1\end{pmatrix}}\cdot{\boldsymbol{P}_1}\cdot\begin{pmatrix}
	{^{W}\boldsymbol{l}}\\1
	\end{pmatrix}
	\\
	\boldsymbol{0}=\liehat{{^{F_2}d_n}\cdot\begin{pmatrix}{^{F_2}\boldsymbol{f}_n}\\1\end{pmatrix}}\cdot{^{F_2}d_n}\cdot\begin{pmatrix}{^{F_2}\boldsymbol{f}_n}\\1\end{pmatrix}&=\liehat{{^{F_2}d_n}\cdot\begin{pmatrix}{^{F_2}\boldsymbol{f}_n}\\1\end{pmatrix}}\cdot{\boldsymbol{P}_2}\cdot\begin{pmatrix}
	{^{W}\boldsymbol{l}}\\1
	\end{pmatrix}
	\end{aligned}
	\end{equation}
	得到:
	\begin{equation}
	\liehat{\begin{pmatrix}{^{F_1}\boldsymbol{f}_m}\\1\end{pmatrix}}\cdot{\boldsymbol{P}_1}\cdot\begin{pmatrix}
	{^{W}\boldsymbol{l}}\\1
	\end{pmatrix}=\boldsymbol{0}
	\qquad
	\liehat{\begin{pmatrix}{^{F_2}\boldsymbol{f}_n}\\1\end{pmatrix}}\cdot{\boldsymbol{P}_2}\cdot\begin{pmatrix}
	{^{W}\boldsymbol{l}}\\1
	\end{pmatrix}=\boldsymbol{0}
	\end{equation}
	
\end{document}




















