\documentclass[12pt, onecolumn]{article}

% 引入相关的包
\usepackage{amsmath, listings, fontspec, geometry, graphicx, ctex, color, subfigure}

% 设定页面的尺寸和比例
\geometry{left = 1.5cm, right = 1.5cm, top = 1.5cm, bottom = 1.5cm}

% 设定两栏之间的间距
\setlength\columnsep{1cm}

% 设定字体,为代码的插入作准备
\newfontfamily\ubuntu{Ubuntu Mono}

% 头部信息
\title{\normf{这是标题}}
\author{\normf{陈烁龙}}
\date{\normf{\today}}

% 代码块的风格设定
\lstset{
	language=C++,
	basicstyle=\scriptsize\ubuntu,
	keywordstyle=\textbf,
	stringstyle=\itshape,
	commentstyle=\itshape,
	numberstyle=\scriptsize\ubuntu,
	showstringspaces=false,
	numbers=left,
	numbersep=8pt,
	tabsize=2,
	frame=single,
	framerule=1pt,
	columns=fullflexible,
	breaklines,
	frame=shadowbox, 
	backgroundcolor=\color[rgb]{0.97,0.97,0.97}
}

% 字体族的定义
\newcommand{\normf}{\kaishu}
\newcommand{\boldf}{\heiti}
\newcommand\keywords[1]{\boldf{关键词:} \normf #1}
\newcommand{\lieal[1]}{\left\lfloor #1 \right\rfloor_\times}

\begin{document}
	
	% 插入头部信息
	\maketitle
	% 换页
	\thispagestyle{empty}
	\clearpage
	
	% 插入目录、图、表并换页
	\tableofcontents
	\listoffigures
	\listoftables
	\setcounter{page}{1}
	% 罗马字母形式的页码
	\pagenumbering{roman}
	\clearpage
	% 从该页开始计数
	\setcounter{page}{1}
	% 阿拉伯数字形式的页码
	\pagenumbering{arabic}
	
	
	\section{\normf{PnP描述}}
	\normf
	对于已知位置的三维点$P_w=\left[ X_w,Y_w,Z_w \right]^T $,其可通过相机的位姿${^{W}_{C}\boldsymbol{T}=\left[ {^{W}_{C}\boldsymbol{R}} |^{W}\boldsymbol{t}_C \right]}$,将其变换到局部相机坐标系下:
	\begin{equation*}
	\boldsymbol{P}_C={^{W}_{C}\boldsymbol{R}^{-1}}\boldsymbol{P}_W-{^{W}_{C}\boldsymbol{R}^{-1}}{^{W}\boldsymbol{t}_C}
	\end{equation*}
	进而将其变换到相机的归一化坐标平面上:
	\begin{equation*}
	\boldsymbol{p}_n=\begin{bmatrix}
	x_n\\y_n
	\end{bmatrix}=\begin{bmatrix}
	X_C/Z_C\\Y_C/Z_C
	\end{bmatrix}
	\end{equation*}
	如果我们测得了对应的特征点图像坐标,并通过一定的相机模型,将其转换到相机的归一化坐标平面上,得到$\boldsymbol{\tilde{p}}_n$,那么可以建立误差函数:
	\begin{equation*}
		\boldsymbol{e}(\boldsymbol{p}_n)=\boldsymbol{p}_n-\boldsymbol{\tilde{p}}_n
	\end{equation*}
	基于误差函数,对待求参数求导(旋转量使用李代数右扰动模型):
	\begin{equation*}
	\begin{cases}
		\begin{aligned}
		\frac{\partial \boldsymbol{e}}{ \delta^{W}_{C}\boldsymbol{R}}&=\frac{\partial \boldsymbol{p}_n}{\partial \boldsymbol{P}_C}\times \frac{\partial \boldsymbol{P}_C}{\partial ^{C}_{W}\boldsymbol{R}}\times \frac{\partial ^{W}_{C}\boldsymbol{R}}{\partial ^{C}_{W}\boldsymbol{R}}
		=\begin{bmatrix}
		1/Z_C&0&-X_C/Z_C^2\\
		0&1/Z_C&-Y_C/Z_C^2
		\end{bmatrix}
		\times \left[ -{^{W}_{C}\boldsymbol{R}^{-1}}\lieal[\boldsymbol{P}_W]\right] \times \left[- ^{W}_{C}\boldsymbol{R}\right] \\
		\frac{\partial \boldsymbol{e}}{\partial ^{W}\boldsymbol{t}_C}&=\frac{\partial \boldsymbol{p}_n}{\partial \boldsymbol{P}_C}\times \frac{\partial \boldsymbol{P}_C}{\partial ^{W}\boldsymbol{t}_C}=\begin{bmatrix}
		1/Z_C&0&-X_C/Z_C^2\\
		0&1/Z_C&-Y_C/Z_C^2
		\end{bmatrix}
		\times \left[- {^{W}_{C}\boldsymbol{R}^{-1}}\right] 
		\end{aligned}
	\end{cases}
	\end{equation*}
	使用高斯-牛顿法求解迭代量:
	\begin{equation*}
		\begin{cases}
			\begin{aligned}
			    \boldsymbol{J}_i&=\begin{bmatrix}
			    \frac{\partial \boldsymbol{e}}{ \delta^{W}_{C}\boldsymbol{R}}&\frac{\partial \boldsymbol{e}}{\partial ^{W}\boldsymbol{t}_C}
			    \end{bmatrix}_{2\times 6} \\
				\boldsymbol{H}_{6\times 6}&=\sum\boldsymbol{J}_i^T\boldsymbol{J}_i\\
				\boldsymbol{g}_{6\times 1}&=\sum- \boldsymbol{J}_i^T\boldsymbol{e}\\
				\boldsymbol{H} \Delta\boldsymbol{x}&=\boldsymbol{g}
			\end{aligned}
		\end{cases}
	\end{equation*}
	
	\section{\normf{四元素姿态与优化}}
	\normf
	对于轴角\footnote{\normf{李代数so3在$R^3$下的表示。}}$\boldsymbol{\theta}=\left[ \theta_x ,\theta_y,\theta_z\right]^T $,其四元素表示为:
	\begin{equation*}
		\boldsymbol{q}=	\frac{\boldsymbol{\theta}}{\theta}\sin\frac{\theta}{2}+\cos\frac{\theta}{2}=\begin{bmatrix}
		q_x&q_y&q_z&q_w
		\end{bmatrix}^T\quad \theta=\sqrt{\theta_x^2+\theta_y^2+\theta_z^2}
	\end{equation*}
	
	那么四元素对轴角的导数为有:
	\begin{equation*}
		\begin{cases}
		\begin{aligned}
			c_0&=q_w/2\quad c_1=q_z/2 \quad c_2=-c_1\\
			c_3&=q_y/2\quad c_4=q_x/2 \quad c_5=-c_4\\
			c_6&=-c_3
		\end{aligned}
		\end{cases}
	\end{equation*}
	\begin{equation*}
	\frac{\partial \boldsymbol{q}}{\partial\boldsymbol{ \theta}}=\boldsymbol{J}=\begin{bmatrix}
	c_0&c_2&c_3\\c_1&c_0&c_5\\c_6&c_4&c_0\\c_5&c_6&c_2
	\end{bmatrix}
	\end{equation*}
	
	
	
	
	
	
	
	
	
	
	
	
	
	
	
	
	
	
	
\end{document}

